\documentclass{sigchi-alternate}

% Use this command to override the default ACM copyright statement (e.g. for preprints). 
% Consult the conference website for the camera-ready copyright statement.
%\toappear{%
%Permission to make digital or hard copies of all or part of this work for personal or classroom use is granted without fee provided that copies are not made or distributed for profit or commercial advantage and that copies bear this notice and the full citation on the first page. Copyrights for components of this work owned by others than the author(s) must be honored. Abstracting with credit is permitted. To copy otherwise, or republish, to post on servers or to redistribute to lists, requires prior specific permission and/or a fee. Request permissions from \href{mailto:permission@acm.org}{Permissions@acm.org}.\\[3pt]
%\textit{CHI 2014}, April 26--May 1, Toronto, ON, Canada.\\
%Copyright is held by the owner/author(s). Publication rights licensed to ACM.
%}
%\submissionversion{CHI 14}

% Load basic packages
\usepackage{balance}  % to better equalize the last page
\usepackage{graphicx} % for EPS, load graphicx instead
\usepackage{url}      % llt: nicely formatted URLs
\usepackage[utf8]{inputenc}

% llt: Define a global style for URLs, rather that the default one
\makeatletter
\def\url@leostyle{%
	\@ifundefined{selectfont}{\def\UrlFont{\sf}}{\def\UrlFont{\small\bf\ttfamily}}}
\makeatother
\urlstyle{leo}

% remove those two lines if you don't want to use biblatex
\usepackage[style=sigchi,backend=biber,doititles]{biblatex}
\addbibresource{references.bib}
%\DeclareBibliographyCategory{ignore}
%\addtocategory{ignore}{AGORAVA}
%\addtocategory{ignore}{CLOUDRAIL}
%\addtocategory{ignore}{SOCIALMEDIA-ABSTRACTIONS}
%\addtocategory{ignore}{ASNE}

% hyperref was already loaded by the documentclass.
% Use nohyperref option to prevent this from happening.
\hypersetup{
	pdftitle={Ingen titel än},
	pdfauthor={LaTeX},
	pdfkeywords={SIGCHI, proceedings, archival format},
	bookmarksnumbered,
	pdfstartview={FitH},
	colorlinks,
	citecolor=black,
	filecolor=black,
	linkcolor=black,
	urlcolor=black,
	breaklinks=true,
}

% create a shortcut to typeset table headings
\newcommand\tabhead[1]{\small\textbf{#1}}

% End of preamble. Here it comes the document.
\begin{document}

% Arabic page numbers for submission. 
% Add this line to eliminate page numbers for the camera ready copy
%\pagestyle{empty}

\title{Ingen titel än}

\numberofauthors{2}
\author{
	\alignauthor{Fredrik Jonsén\\
		\affaddr{Linköping University}\\
		\affaddr{Linköping, Sweden}\\
		\email{frejo105@student.liu.se}}
	\alignauthor{Alexander Stolpe\\
		\affaddr{Linköping University}\\
		\affaddr{Norrköping, Sweden}\\
		\email{alest170@student.liu.se}}
}

% Teaser figure can go here
%\teaser{
%  \centering
%  \includegraphics{Figure1}
%  \caption{Teaser Image}
%  \label{fig:teaser}
%}

\maketitle

\begin{abstract}
Coming later
\end{abstract}

\section{Introduction}
Today a lot of people are using social media in one way or the other and it is estimated that there will be around 2.67 billion social media
users around the globe by 2018\autocite{STATISTA_SN_WORLD_USERS}. Most of these social networks have released Application Programming Interfaces (APIs)
which developers wish to integrate these networks into their software.

\subsection{Motivation}
Social network usage is growing and has gone from 0.97 billion users in 2010 to 2.14 billion in 2015.\autocite{STATISTA_SN_WORLD_USERS}. This would
account for approximately 29\% of the earth's population in 2015, which was 7.347 billion 2015\autocite{WORLD_BANK_POPULATION}. It is worth noting
that this counts created user accounts and not unique users, one person can have multiple accounts over multiple networks, and accounts may not
belong to an actual person, but rather companies or bots.

Because of the currently high, and still growing, number of social media users we find it highly likely that we will see an increasing number of
applications that involve social media in their software in one way or the other. Out of all the social networks existing today there are twenty
that have more than 100 million active accounts\autocite{STATISTA_LEADING_SOCIAL_NETWORKS}. This means that if one would want to create an application
that involves a lot of social networks we will have to do a lot of work just to implement all of these into our system.

\subsection{Purpose}
Because of this we see a need for a way to combine these social network APIs in some way to save development
time and reduce the amount of duplicate code written in software. The purpose for this project is to create a library 
that combines the APIs in a modular fashion, where each API serves as a module in order to simplify adding new networks,
and make it easier to involve social media in software. 

A software library by definition is a set of pre-written code that a developer might add to a project in order to add more functionality or to ease the 
development process\autocite{TLDP_LIBRARY_DEFINITION}.

\section{Research Question}
Is it possible to create a maintainable modular library for Social Network APIs?

\subsection{Definitions}
For some terms, it may be necessary to give a more specific definition.

\subsubsection{Maintainability}
Maintainability can be defined as the simplicity with which defects can be corrected and the library can be extended or modified to support future 
requirements.\autocite{5733835}. This can generally be measured by certain attributes, such as unit test coverage, lines of code, cyclomatic complexity,
and language specific code styles\autocite{SONARQUBE_MAINTAINABILITY_DEFINITION}. For the evaluation of our library, we will be making using of the third
party tool SonarQube\footnote{https://www.sonarqube.org/}.

\subsubsection{Modularity}
For the purpose of our study, we define modularity as the extent to which a program can be divided into modules, where\autocite{Kiczales:2005:APM:1062455.1062482}:
\begin{itemize}
	\item Each module has a well-defined interface that describes how the system can interact with it.
	\item Changes to the underlying functionality of a module does not affect the modules interface.
	\item Modules can be put together with other modules in different ways to make a complete program
\end{itemize}
In our study modularity will mostly affect how easy it will be to integrate new social media APIs in the future.

\section{Limitations}
Because there are so many social networks existing today and we have a finite amount of time to complete this
project we will focus on a smaller set of APIs to implement into our library.  We have set up a few criteria 
for the APIs so we can find suitable candidates for our library, which are:
\begin{itemize}
	\item The social network must be one of the 22 most popular\autocite{STATISTA_LEADING_SOCIAL_NETWORKS}.
	\item It must be relevant for our geographical location, in this case Europe.
\end{itemize}
Other than this we will judge the API itself subjectivly by how good its documentation is, its functionality and its ease-of-use. 

\section{Theory}
\subsection{Code Reuse}
A lot of time and money can be saved by reusing code, of which libraries are one form. Although there are
some issues with using libraries, the gains from avoiding reinventing the wheel makes writing and using
libraries a common practice, in particular in open source software\autocite{2998479020080101}.

Writing code which can be easily reused requires a deeper analysis of the problem domain, which may increase the cost and time required compared 
to developing the same code without reuse in mind, but can drastically decrease the cost of developing systems in the future where the code can 
be reused \autocite{lim1994effects}. This cost reduction is apparent foremost in terms of direct cost of development, but also in time-to-market, 
which can be argued to be even more important in the long term\autocite{griss1993software}.

With code reuse there are also several potential issues which have to be kept in mind, both when it comes to the implementation itself but also 
when it comes to using the implementation. Backwards compatibility between versions is a major topic in itself\autocite{raemaekers2012measuring}. There 
is also the risk of a library being abandoned by it’s maintainer. This is especially true for proprietary libraries, where the source code may not 
be available. It this case, the library might have to be replaced, making all the effort to use the library wasted.

\subsection{Library Interface Design}
When designing our library we will want keep several things in mind. We want to design the library so it’s easy and straightforward to use for a 
developer. As Henning points out in his article\autocite{Henning:2007:ADM:1255421.1255422}, about design of APIs, it is very easy to create a bad one, 
but very hard to create an API that feels natural and easy to work with. APIs, as we know, are a kind of interface for a program to gain access to 
another program without direct access, and can be compared to the interface of our library.

Henning continues to discuss guidelines for how an API should generally be designed. What feels most relevant to our work is how 
he describes how the APIs should be designed from the perspective of a user, because when it’s done from the implementer's point of view the 
needs of the user are often forgotten. It’s usually best to document first, because when it is done after the implementation the programmer, who 
wrote the functionality, will usually just dictate what he did, rather than make it is obvious enough for others who are not as familiar with the code.

\subsection{Social Media APIs}
As mentioned, one of the biggest issues with using libraries is the risk of the project being abandoned. This risk increases significantly when the library
itself uses APIs which integrate oft-changing platforms, such as the case of Social Media. The rate of change differs between networks, in some cases on average
three times a year\footnote{https://developers.facebook.com/docs/apps/changelog}, in other cases several times a month\footnote{https://www.hitchhq.com/twitter/activities},
although the impact of the changes varies greatly. At best, a lack of active development simply means missing out on new functionality. In other cases, such as 
unfixed security vulnerabilities\footnote{https://github.com/gorbin/ASNE/issues/107}, may render the library unusable.

\subsection{Similiar work}
Several similar works, Agorava\footnote{http://www.agorava.org/}, ASNE\footnote{https://github.com/gorbin/ASNE} and 
SocialMedia Abstractions\footnote{https://github.com/socialsensor/socialmedia-abstractions} serve the same or similar purposes, but all were abandoned before reaching a 
stable release. In the case of ASNE, the project was abandoned explicitly due to a lack of free time. This shows the issue of a library having only a single maintainer, 
as the project risks being abandoned by it's developer as soon the project is no longer a priority. The others have no stated reason for the lack of continued development.
There are also commercial services\footnote{https://cloudrail.com} which provide this functionality for some popular APIs, but charge money to sign up 
and use. As it is proprietary it's inner workings are completely opaque, and thus will not be examined by this study.

\subsubsection{Encountered issues}
Despite being abandoned, we hope to still learn from the problems the similar libraries encountered and solved. For this, we looked into each project's issue
tracking (where available) and commits. This was somewhat complicated in that one project, SocialMedia Abstractions, simply had not used issue tracking. 
Another project, ASNE did use issue tracking, but much of the discussion regarding individual issues was largely in Russian, making it unusable in our case. 
The issues, despite the name, did not always regard bugs. In the vast majority of cases, issued stemmed from users misunderstanding the library documentation,
requesting features, or suggesting refactoring of code to increase maintainability. In the case of documentation misunderstandings, these were often
solved by simply adding examples. There was also a noticeable difference in the amount of issues pertaining bugs in ASNE, which included no automatic tests,
compared to Agorava, which includes a large amount of automatic tests, and had almost no issues regarding logical errors, despite having a much larger code base.

% Balancing columns in a ref list is a bit of a pain because you
% either use a hack like flushend or balance, or manually insert
% a column break.  http://www.tex.ac.uk/cgi-bin/texfaq2html?label=balance
% multicols doesn't work because we're already in two-column mode,
% and flushend isn't awesome, so I choose balance.  See this
% for more info: http://cs.brown.edu/system/software/latex/doc/balance.pdf
%
% Note that in a perfect world balance wants to be in the first
% column of the last page.
%
% If balance doesn't work for you, you can remove that and
% hard-code a column break into the bbl file right before you
% submit:
%
% http://stackoverflow.com/questions/2149854/how-to-manually-equalize-columns-
% in-an-ieee-paper-if-using-bibtex
%
% Or, just remove \balance and give up on balancing the last page.
%
\balance

% If you want to use smaller typesetting for the reference list,
% uncomment the following line:
% \small

% If you don't want to use biblatex uncomment the following two lines and remove the one after
%\bibliographystyle{acm-sigchi}
%\bibliography{sample}
\printbibliography[notcategory=ignore]

\end{document}
