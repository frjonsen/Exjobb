% !TeX spellcheck = sv_SE
\documentclass{sigchi}

\pagenumbering{arabic}

\usepackage{balance}
\usepackage{graphics}
\usepackage[T1]{fontenc}
\usepackage{txfonts}
\usepackage[pdflang={sv-SE},pdftex]{hyperref}
\usepackage{color}
\usepackage{booktabs}
\usepackage{textcomp}


\usepackage{microtype}
\usepackage[all]{hypcap}    % Fixes bug in hyperref caption linking
\usepackage{ccicons}
\usepackage[utf8]{inputenc}

\def\plaintitle{Ingen titel än}
\def\plainauthor{Fredrik Jonsén, Alexander Stolpe}

\makeatletter
\def\url@leostyle{
	\@ifundefined{selectfont}{
		\def\UrlFont{\sf}
	}{
		\def\UrlFont{\small\bf\ttfamily}
}}
\makeatother
\urlstyle{leo}

\def\pprw{8.5in}
\def\pprh{11in}
\special{papersize=\pprw,\pprh}
\setlength{\paperwidth}{\pprw}
\setlength{\paperheight}{\pprh}
\setlength{\pdfpagewidth}{\pprw}
\setlength{\pdfpageheight}{\pprh}

\definecolor{linkColor}{RGB}{6,125,233}
\hypersetup{
	pdfauthor={\plainauthor},
	pdfdisplaydoctitle=true,
	bookmarksnumbered,
	colorlinks,
	citecolor=black,
	filecolor=black,
	linkcolor=black,
	urlcolor=linkColor,
	breaklinks=true,
}

\begin{document}
\title{\plaintitle}
\numberofauthors{2}
\author{
	\alignauthor{Fredrik Jonsén\\
		\affaddr{Linköpings universitet}\\
		\affaddr{Linköping, Sverige}\\
		\email{frejo105@student.liu.se}}
	\alignauthor{Alexander Stolpe\\
		\affaddr{Linköpings universitet}\\
		\affaddr{Linköping, Sverige}\\
		\email{alest170@student.liu.se}}
}

\maketitle

\begin{abstract}
I dagsläget så använder sig väldigt många människor utav sociala medier, och antalet
användare ökar kraftigt varje år\cite{STATISTA_SN_WORLD_USERS}. Detta gör det allt mer
attraktivt för utvecklare att involvera sociala nätverk i sin mjukvara. Eftersom plattformarna
skiljer sig åt i funktionalitet och användarinteraktion kan det vara relevant att studera hur
man på bästa sätt binder samman flera olika av dessa. Detta för att nå ut till så många som
möjligt, samt möjligheten att använda den  plattform som bäst lämpar sig för syftet. För att
undersöka vilken lösning som anses lämplig och i vilken mån detta kan vara till praktisk nytta
så har vi gjort en studie utav de stora aktörerna inom sociala media. Därefter implementerade vi
en mindre prototyp för att visa på hur en lösning av detta kan se ut.
\end{abstract}

\section{Inledning}
\subsection{Motivering}
Sociala nätverk ökade mellan 2010 till 2015 från 0,97 miljarder till 2,14 miljarder användare. Som referenspunkt motsvarar detta 29\% av Jordens befolkning år 2015 (7,347 miljarder)\cite{WORLD_BANK_POPULATION}. 
År 2018 förväntas antal användare ligga runt 2,67 miljarder\cite{STATISTA_SN_WORLD_USERS}. Här räknas dock
antalet konton, och inte unika användare. De största aktörerna på marknaden var, i januari 2017, Facebook, WhatsApp, Facebook Messenger, QQ, samt WeChat\cite{STATISTA_LEADING_SOCIAL_NETWORKS}.

Då antalet människor som använder sociala medier ökar så blir behovet allt större att nå ut till alla dessa. 
För att nå ut till så många som möjligt räcker det inte med att enbart rikta sig mot en av dessa. De 
flesta av de stora plattformarna har i dagsläget öppna API:er som man som utvecklare kan använda sig av 
för att utveckla program och applikationer mot.

\subsection{Syfte}
För att ta reda på vilka av de stora plattformarna som hade den önskvärda funktionaliteten, 
i sina respektive API:er, så gjorde vi inledningsvis en studie av dessa.

För att sedan testa hur det i praktiken kan se ut när man skapar ett ramverk som binder 
ihop flera sociala nätverk så har vi utvecklat en prototyp för mobilapplikationen Blixtvakt, 
en applikation som varnar användare om blixtnedslag någonstans innanför ett valt område. Tanken 
bakom ramverket vi byggt är att försöka utnyttja sociala nätverk som en kommunikationskanal mellan 
appen och användare, exempelvis för att skicka ut information, samt mellan användare.

\section{Frågeställning}
Hur utvecklar man ett ramverk för att binda samman olika sociala nätverk utifrån följande faktorer:
\begin{itemize}
	\item Hur når man ut till så många som möjligt?
	\item Vilka av plattformarna lämpar sig för direkt, respektive indirekt, kommunikation?
	\item Föredrar användare att bli kontaktade indirekt eller direkt? 
\end{itemize}
\section{Avgränsningar}
Då det finns väldigt många sociala medier måste vi välja en mindre mängd att inrikta oss mot 
när vi utvecklar vårt program. Vi måste även i vårt arbete fokusera på vad som lämpar sig bäst 
för att integrera med applikationen Blixtvakt. För att välja dessa så har vi valt att utgå 
från dessa kriterier:
\begin{itemize}
	\item Plattformen måste vara en av de 22 största\cite{STATISTA_LEADING_SOCIAL_NETWORKS}.
	\item Plattformen måste vara relevant att länka samman men Blixtvakt.
	\item Det måste finnas bra stöd i plattformens API.
\end{itemize}
\balance{}

\bibliographystyle{SIGCHI-Reference-Format}
\bibliography{references}

\end{document}